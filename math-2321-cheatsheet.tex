\documentclass[10pt,landscape,letterpaper]{article} % Copyright (c) 2020 Brian Schubert
\usepackage{multicol}
\usepackage[landscape]{geometry}
\usepackage{amsmath,amsfonts,amssymb}
\usepackage{fancyhdr}
\usepackage{enumitem}
\usepackage{tabularx}
\usepackage{titlesec}
\usepackage{array}

%
% Styling inspired by tex.stackexchange user "Dror" in this posting:
% https://tex.stackexchange.com/questions/8827/preparing-cheat-sheets
%

\thispagestyle{fancy}
\cfoot{\footnotesize Copyright \textcopyright \ 2020 Brian Schubert}
\rfoot{\footnotesize Revised 2020-02-12}
\lhead{MATH 2321\\Calculus 3 for Science and Engineering}
\rhead{Spring 2020 \\ Dr. Sumi Seo}
%\chead{\large \underline{Midterm Exam ``Cheat Sheet"}}
\renewcommand{\headrulewidth}{0pt}
\setlength{\voffset}{-4pt}

\geometry{top=0.7in,left=.4in,right=.4in,bottom=.6in}


% Redefine section commands to use less space
% NOTE - still to be tweak to maximize space with reasonable readbability
\makeatletter
\renewcommand{\section}{\@startsection{section}{1}{0mm}%
                                {-1ex plus -.5ex minus -.2ex}%
                                {0.5ex plus .2ex}%x
                                {\normalfont\large\bfseries}}
\renewcommand{\subsection}{\@startsection{subsection}{2}{0mm}%
                                {-1explus -.5ex minus -.2ex}%
                                {0.5ex plus .2ex}%
                                {\normalfont\normalsize\bfseries}}
% Custom squishing for \paragraph
\renewcommand\paragraph{\@startsection{paragraph}{4}{\z@}%
    {0.5ex \@plus1ex \@minus.2ex}%
    {-1em}%
    {\normalfont\small\bfseries}}

% Don't print section numbers
\setcounter{secnumdepth}{0}



\setlength{\parindent}{0pt}
\setlength{\parskip}{0pt plus 0.5ex}

% Notation Definitions
\newcommand{\dd}[1]{\mathrm{d}#1}
\newcommand{\diffop}[2][]{\frac{\dd#1}{\dd #2}}
\newcommand{\pdiffop}[2][]{\frac{\partial #1}{\partial #2}}
\newcommand{\matr}[1]{\mathbf{#1}}
\newcommand{\bvec}[1]{\mathbf{#1}}
\newcommand{\vecspace}[1]{\mathbb{#1}}
\newcommand{\transp}{^\mathrm{T}}

\newcommand\cheatsheetmargin{0.2cm}

% -----------------------------------------------------------------------

\begin{document}
\raggedright
\footnotesize
\begin{multicols}{3}


% multicol parameters
% These lengths are set only within the two main columns
%\setlength{\columnseprule}{0.25pt}
\setlength{\premulticols}{1pt}
\setlength{\postmulticols}{1pt}
\setlength{\multicolsep}{1pt}
\setlength{\columnsep}{2pt}

%\begin{center}
%     \Large{\underline{Preliminaries}} \\
%\end{center}
\section{Geometry in $\mathbb R^n$}
\paragraph{Orthogonal Projection \& Vector Product Identities} \begin{gather*}
    \operatorname{proj}_\bvec{v} \bvec{F} = \left(\frac{\bvec F \cdot \bvec v}{\bvec v \cdot \bvec v}\right) \bvec v = \left(\frac{\bvec F \cdot \bvec v}{\|\bvec v\|^2}\right) \bvec v \\
    \bvec v \cdot \bvec w = \|\bvec v\|\|\bvec w\| \cos\theta = \|\bvec v\| \|\operatorname{proj}_\bvec{v} \bvec w\|  = \|\bvec w\| \|\operatorname{proj}_\bvec{w} \bvec v \| \\
    \bvec v \times \bvec w = -( \bvec w \times \bvec v), \quad \| \bvec v \times \bvec w\| = \|\bvec v\|\|\bvec w\| \sin\theta\\
    ( \bvec v \times \bvec w) \cdot \bvec v = ( \bvec v \times \bvec w) \cdot \bvec w = \bvec 0
\end{gather*}
\subsection{Planes} A plane in $\mathbb R^m$ which contains the points $\bvec p=\langle p_1,\ldots,p_m \rangle$ and normal to the vector $\bvec n = \langle n_1,\ldots,n_m \rangle$ is given by
\begin{gather*}
\bvec{n}\cdot (\bvec x - \bvec p) = \bvec 0 \iff n_1(x_1 - p_1) + \cdots + n_m(x_m - p_m) = 0\tag{basic form}\\
n_1 x_1 + \cdots + n_m x_m + d = 0, \quad d=-n_1 p_1 - \cdots -n_mp_m \tag{standard form}
\end{gather*}
\subsection{Lines}A line $\ell$ in $\mathbb R^n$ containing the point $\bvec p$ and is parallel to the vector $\bvec v$ is given by
\begin{gather*}
    \bvec x(t) = \bvec p + t\bvec v \tag{Vector Equation}\\
    {x}_1 = p_1 + v_1 t, \ \ldots,\ {x}_n = p_n + v_n t \tag{Parametric Equation}\\
    \frac{x_1 - p_1}{v_1} = \cdots = \frac{x_n - p_n}{v_n} \tag{Symmatric Equation}
\end{gather*}
\paragraph{Tangent Line} Given $\bvec p: \mathbb R \to\mathbb R^n$, the tangent line $L$ at $t=t_0$ is parameterized by $L(r) = \bvec p(t_0) + r \bvec p'(t_0)$.

\section{Basic Topology of $\vecspace{R}^n$}
\paragraph{Open Ball} The open ball centered at $\bvec p \in \vecspace R^n$ of radius $r$ is
\begin{equation*}
    B_r^n (\bvec p) = \left\{\bvec x \in \vecspace{R}^n \ \vert \ \operatorname{dist} (\bvec x, \bvec p) < r \right\}
\end{equation*}
\paragraph{Open Set} $E\subseteq \mathbb R^n$ is an open set if for every point $\bvec p \in E$, there exists an $r>0$ such that $B^n_r \subseteq E$.
\paragraph{Boundary Point} A point $\bvec p \in E$ is a boundary point if every open ball around $\bvec p$ contains both a point in $E$ and a point not in $E$.

\paragraph{Closed Set }$E\subseteq \mathbb R^n$ is closed is it contains all of its boundary points.

\paragraph{Bounded Set} $E$ is bounded if there exists $r>0$ such that $B_r^n(\bvec 0)$ contains $E$.

\paragraph{Compact Set} $E\subseteq \mathbb R^n$ is compact is it is both closed and bounded.

\section{Linearization \& Linear Approximation}
For $f: \vecspace{R}^n \to \vecspace{R}$, the linearization of $f$ about point $\bvec{p} \in \vecspace{R}^n$ is
\begin{equation*}
L_f(\bvec{x};\, \bvec{p}) = f(\bvec{p}) + \nabla f(\bvec p) \cdot (\bvec x - \bvec p)
\end{equation*}
\paragraph{Linear Approx.} $f(\bvec x) \approx L(\bvec{x};\, \bvec p)$ when $\bvec{x}$ near $\bvec p$.

\section{Multivariable Derivatives}
Given $f: \vecspace{R}^n \to \vecspace{R}$ where $f = f(\bvec x)$, the partial derivative of $f$ with respect to the $i$th component of $\bvec x$ is given by
\begin{equation*}
    f_{x_i}(\bvec p) = \left.\frac{\partial f}{\partial x_i}\right|_\bvec{p}
        = \lim\limits_{h \to 0} \frac{f(\bvec p + h\hat{\mathbf{e}}_i)-f(\bvec p)}{h}
\end{equation*}

\paragraph{Gradient} For the function $f$ above, the gradient of $f$, $\nabla f: \vecspace{R}^2 \to \vecspace{R}^2$ is given by
\begin{equation*}
    \nabla f = \left\langle \pdiffop[f]{x_1}, \pdiffop[f]{x_2}, \ldots, \pdiffop[f]{x_n}  \right\rangle
\end{equation*}

\paragraph{High Order Partial Derivatives}
If $f_{xy}$ exists and is continuous at $\bvec p$, then $f_{xy}(\bvec p) = f_{yx}(\bvec p)$.

\section{Differentiation Rules}
Let $f,g: \vecspace{R}^n \to \vecspace{R}$ and $a,b \in \vecspace{R}$.
\begin{align*}
    &\nabla( af + bg)  &=& a \nabla f + b \nabla g \tag{Linearity} \\
    &\nabla(fg) &=& f \nabla g + g \nabla f \tag{Product Rule} \\
    &\nabla(f/g) &=&\left(g\nabla f -  f \nabla g\right) / g^2 \tag{Quotient Rule}\\
    &\nabla (f^\alpha) &=& \alpha f^{\alpha - 1} \nabla f \tag{Power Rule}
\end{align*}
\paragraph{Chain Rule} Let $f: \vecspace{R}^n \to \vecspace{R}$, $f=f(\bvec{x})$, and $\bvec x: \vecspace{R}^p \to \vecspace{R}^n$ where $\bvec{x} = \bvec{x}(\bvec{t})$. Then
\begin{equation*}
    \frac{\partial f}{\partial t_i} = \nabla f \cdot \frac{\partial \bvec x}{\partial t_i}
\end{equation*}
E.g. for $f:\mathbb R^3 \to \mathbb R$, $\bvec x = \bvec x(s,t)$, \begin{equation*}
    f_s = f_x x_s + f_y y_s + f_z z_s
\end{equation*}


\section{Directional Derivative}
Consider $f: \vecspace{R}^n \to \vecspace{R}$ and $\bvec p, \bvec u \in \vecspace{R}^n$ with $\|\bvec u\|=1$. The directional derivative of $f$ in the direction of $\bvec u$ (or, with respect to $\bvec u$) at point $\bvec p$ is given by
\begin{equation*}
    D_\bvec{u} f(\bvec p) = \nabla f(\bvec p) \cdot \bvec u, \qquad D_{\bvec{e}_i} f = \pdiffop[f]{x_i}
\end{equation*}
Note that the directional derivative attains it maximum, $\|\nabla f(\bvec p)\|$, when $\bvec u$ is the unit vector in the direction of the gradient of $f$.

\section{Level Sets}
For a function $f: E \to \vecspace{R}$ where $E \subseteq \vecspace{R}^n$ and $k \in \vecspace R$, the level set where $f=k$  (or, the set containing $\bvec p$ where $f(\bvec p)=k$) is the set
\begin{equation*}
\left\{ \bvec{x} \in E \ \big\vert \ f(\bvec x) = k \right\}
\end{equation*}
\begin{itemize}[leftmargin=*,noitemsep]
    \item When $f:\mathbb R^2 \to \mathbb R$, the level set is called the \underline{``level curve"}.
    \item When $f:\mathbb R^3 \to \mathbb R$, the level set is called the \underline{``level surface"}.
    \item In general, the graph of a level surface of $f$ has one less dimensions that the graph of $f$.
    \item The gradient of $f$ is $\perp$ to the tangent line of a level set.
\end{itemize}

\paragraph{Tangent Set} The tangent set to the level set where $F=F(\bvec p)$, $\bvec p \in \mathbb R^n$, is given by \begin{equation*}
    \left\{  \bvec{x} \in \mathbb R^n \ \big\vert \ \nabla F(\bvec p) \cdot (\bvec x - \bvec p) = \bvec 0 \right\}
\end{equation*}
When $F: \begin{array}{c}\mathbb R^2 \\ \mathbb R^3\end{array}\to\mathbb R$, the tangent set is call the \underline{tangent}  $\begin{array}{c}\text{\underline{line}} \\ \text{\underline{plane}}\end{array}$

\paragraph{Critical Point} For $F: \mathbb R^n \to \mathbb R$, a critical point of $F$ is a point $\bvec p$ where either 1) $\nabla F(\bvec p) = \bvec 0$ or 2) F is not differentiable.

\section{Parameterizing Surfaces}
We frequently see surfaces as\begin{enumerate}[noitemsep]
    \item Graphs of functions $\mathbb R^2 \to \mathbb R$: z = f(x,y)
    \item Level surfaces of functions $\mathbb R^3 \to \mathbb R$: F(x,y,z) = c
\end{enumerate}
\paragraph{General Notion} Produce a function $\bvec r : D \to \mathbb R^n$ where $D \subset \mathbb R^{n-1}$ and refer to the image as the parameterized surface.
\paragraph{Regular Surfaces} A parameterization $\bvec r: D \to \mathbb R^3$, $D\subset \mathbb R^2$ is said to be (locally) regular at $\bvec p$ if 1) it is continuously differentiable (i.e. is $C^1$) at $\bvec  p$, and 2) all of the first order partial derivations of $\bvec r$ are linearly independent at $\bvec p$.

\subsection{Tangent Planes to Surfaces}
Consider the parameterization $\bvec r: D \to \mathbb R^3$, $D\subset \mathbb R^2$, with $\bvec p = \bvec{r}(u_0, v_0)$. The tangent plane to the surface at point $\bvec p$ is described by either
\begin{equation*}
(x,y,z) = \bvec{r}(u_0, v_0) + a\bvec{r}_u(u_0, v_0) + b\bvec{r}_v(u_0, v_0), \quad a,b\in\mathbb R
\end{equation*}
or, by letting $\bvec n = \bvec{r}_u \times \bvec{r}_v$,
\begin{equation*}
\bvec n \cdot (\bvec x - \bvec p) = \bvec 0
\end{equation*}

\section{Local Extrema}
\paragraph{Saddle Point} For $f: E \to \mathbb R$, $E\subset \mathbb R^n$, a saddle point is critical point $\bvec p$ at which $f$ is differentiable and $\nabla f(\bvec p) = \bvec 0$, which at which $f$ \emph{does not} attain a local extreme value.
\paragraph{Hessian Matrix} Suppose $f=f(x,y)$ is $C^2$ and $\bvec p \in \mathbb R^2$. The Hessian matrix of $f$ at $\bvec p$ is the matrix \begin{equation*}
    \bvec H(\bvec p) = \left.\begin{bmatrix}
        f_{xx} & f_{xy} \\ f_{yx} & f_{yy}
    \end{bmatrix}\right|_\bvec{p}
\end{equation*}
\paragraph{Non-Degeneracy} A point $\bvec p$ is a non-degenerate critical point if $\nabla f(\bvec p) = \bvec 0$ and $\det \bvec{H}(\bvec p) \neq 0$
\subsection{2nd Derivative Test}
Suppose $f=f(x,y)$ is $C^2$ and $\bvec p$ is a non-degenerate point of $f$. Then,
\begin{enumerate}[noitemsep]
    \item If $\det \bvec{H}(\bvec p) > 0$ and $f_{xx}(\bvec p) > 0$, $f$ attains a local min.
    \item If $\det \bvec{H}(\bvec p) > 0$ and $f_{xx}(\bvec p) < 0$, $f$ attains a local max.
    \item If $\det \bvec{H}(\bvec p) < 0$, $f$ has a saddle point at $\bvec p$
\end{enumerate}





\end{multicols}
\end{document}