\documentclass[10pt,landscape,letterpaper]{article} % Copyright (c) 2020 Brian Schubert
\usepackage{multicol}
\usepackage[landscape]{geometry}
\usepackage{amsmath,amsfonts,amssymb}
\usepackage{mathtools}
\usepackage{fancyhdr}
\usepackage{enumitem}
\usepackage{tabularx}
\usepackage{titlesec}
\usepackage{array}

%
% Styling inspired by tex.stackexchange user "Dror" in this posting:
% https://tex.stackexchange.com/questions/8827/preparing-cheat-sheets
%

\thispagestyle{fancy}
\cfoot{\footnotesize Copyright \textcopyright \ 2020 Brian Schubert}
\rfoot{\footnotesize Revised 2020-02-12}
\lhead{MATH 2321\\Calculus 3 for Science and Engineering}
\rhead{Spring 2020 \\ Dr. Sumi Seo}
%\chead{\large \underline{Midterm Exam ``Cheat Sheet"}}
\renewcommand{\headrulewidth}{0pt}
\setlength{\voffset}{-4pt}

\geometry{top=0.7in,left=.4in,right=.4in,bottom=.6in}


% Redefine section commands to use less space
% NOTE - still to be tweak to maximize space with reasonable readbability
\makeatletter
\renewcommand{\section}{\@startsection{section}{1}{0mm}%
                                {-1ex plus -.5ex minus -.2ex}%
                                {0.5ex plus .2ex}%x
                                {\normalfont\large\bfseries}}
\renewcommand{\subsection}{\@startsection{subsection}{2}{0mm}%
                                {-1explus -.5ex minus -.2ex}%
                                {0.5ex plus .2ex}%
                                {\normalfont\normalsize\bfseries}}
% Custom squishing for \paragraph
\renewcommand\paragraph{\@startsection{paragraph}{4}{\z@}%
    {0.5ex \@plus1ex \@minus.2ex}%
    {-1em}%
    {\normalfont\small\bfseries}}

% Don't print section numbers
\setcounter{secnumdepth}{0}



\setlength{\parindent}{0pt}
\setlength{\parskip}{0pt plus 0.5ex}

% Notation Definitions
\newcommand{\dd}[1]{\mathrm{d}#1}
\newcommand{\diffop}[2][]{\frac{\dd#1}{\dd #2}}
\newcommand{\pdiffop}[2][]{\frac{\partial #1}{\partial #2}}
\newcommand{\matr}[1]{\mathbf{#1}}
\newcommand{\bvec}[1]{\mathbf{#1}}
\newcommand{\vecspace}[1]{\mathbb{#1}}
\newcommand{\transp}{^\mathrm{T}}

\newcommand\cheatsheetmargin{0.2cm}

% -----------------------------------------------------------------------

\begin{document}
\raggedright
\footnotesize
\begin{multicols}{3}


% multicol parameters
% These lengths are set only within the two main columns
%\setlength{\columnseprule}{0.25pt}
\setlength{\premulticols}{1pt}
\setlength{\postmulticols}{1pt}
\setlength{\multicolsep}{1pt}
\setlength{\columnsep}{2pt}

%\begin{center}
%     \Large{\underline{Preliminaries}} \\
%\end{center}
\section{Geometry in $\mathbb R^n$}
\paragraph{Orthogonal Projection \& Vector Product Identities} \begin{gather*}
    \operatorname{proj}_\bvec{v} \bvec{F} = \left(\frac{\bvec F \cdot \bvec v}{\bvec v \cdot \bvec v}\right) \bvec v = \left(\frac{\bvec F \cdot \bvec v}{\|\bvec v\|^2}\right) \bvec v \\
    \bvec v \cdot \bvec w = \|\bvec v\|\|\bvec w\| \cos\theta = \|\bvec v\| \|\operatorname{proj}_\bvec{v} \bvec w\|  = \|\bvec w\| \|\operatorname{proj}_\bvec{w} \bvec v \| \\
    \bvec v \times \bvec w = -( \bvec w \times \bvec v), \quad \| \bvec v \times \bvec w\| = \|\bvec v\|\|\bvec w\| \sin\theta\\
    ( \bvec v \times \bvec w) \cdot \bvec v = ( \bvec v \times \bvec w) \cdot \bvec w = \bvec 0
\end{gather*}
\subsection{Planes} A plane in $\mathbb R^m$ which contains the points $\bvec p=\langle p_1,\ldots,p_m \rangle$ and normal to the vector $\bvec n = \langle n_1,\ldots,n_m \rangle$ is given by
\begin{gather*}
\bvec{n}\cdot (\bvec x - \bvec p) = \bvec 0 \iff n_1(x_1 - p_1) + \cdots + n_m(x_m - p_m) = 0\tag{basic form}\\
n_1 x_1 + \cdots + n_m x_m + d = 0, \quad d=-n_1 p_1 - \cdots -n_mp_m \tag{standard form}
\end{gather*}
\subsection{Lines}A line $\ell$ in $\mathbb R^n$ containing the point $\bvec p$ and is parallel to the vector $\bvec v$ is given by
\begin{gather*}
    \bvec x(t) = \bvec p + t\bvec v \tag{Vector Equation}\\
    {x}_1 = p_1 + v_1 t, \ \ldots,\ {x}_n = p_n + v_n t \tag{Parametric Equation}\\
    \frac{x_1 - p_1}{v_1} = \cdots = \frac{x_n - p_n}{v_n} \tag{Symmatric Equation}
\end{gather*}
\paragraph{Tangent Line} Given $\bvec p: \mathbb R \to\mathbb R^n$, the tangent line $L$ at $t=t_0$ is parameterized by $L(r) = \bvec p(t_0) + r \bvec p'(t_0)$.

\section{Basic Topology of $\vecspace{R}^n$}
\paragraph{Open Ball} The open ball centered at $\bvec p \in \vecspace R^n$ of radius $r$ is
\begin{equation*}
    B_r^n (\bvec p) = \left\{\bvec x \in \vecspace{R}^n \ \big\vert \ \operatorname{dist} (\bvec x, \bvec p) < r \right\}
\end{equation*}
\paragraph{Open Set} $E\subseteq \mathbb R^n$ is an open set if for every point $\bvec p \in E$, there exists an $r>0$ such that $B^n_r \subseteq E$.
\paragraph{Boundary Point} A point $\bvec p \in E$ is a boundary point if every open ball around $\bvec p$ contains both a point in $E$ and a point not in $E$.

\paragraph{Closed Set }$E\subseteq \mathbb R^n$ is closed is it contains all of its boundary points.

\paragraph{Bounded Set} $E$ is bounded if there exists $r>0$ such that $B_r^n(\bvec 0)$ contains $E$.

\paragraph{Compact Set} $E\subseteq \mathbb R^n$ is compact is it is both closed and bounded.

\section{Linearization \& Linear Approximation}
For $f: \vecspace{R}^n \to \vecspace{R}$, the linearization of $f$ about point $\bvec{p} \in \vecspace{R}^n$ is
\begin{equation*}
L_f(\bvec{x};\, \bvec{p}) = f(\bvec{p}) + \nabla f(\bvec p) \cdot (\bvec x - \bvec p)
\end{equation*}
\paragraph{Linear Approx.} $f(\bvec x) \approx L(\bvec{x};\, \bvec p)$ when $\bvec{x}$ near $\bvec p$.

\section{Multivariable Derivatives}
Given $f: \vecspace{R}^n \to \vecspace{R}$ where $f = f(\bvec x)$, the partial derivative of $f$ with respect to the $i$th component of $\bvec x$ is given by
\begin{equation*}
    f_{x_i}(\bvec p) = \left.\frac{\partial f}{\partial x_i}\right|_\bvec{p}
        = \lim\limits_{h \to 0} \frac{f(\bvec p + h\hat{\mathbf{e}}_i)-f(\bvec p)}{h}
\end{equation*}

\paragraph{Gradient} For the function $f$ above, the gradient of $f$, $\nabla f: \vecspace{R}^2 \to \vecspace{R}^2$ is given by
\begin{equation*}
    \nabla f = \left\langle \pdiffop[f]{x_1}, \pdiffop[f]{x_2}, \ldots, \pdiffop[f]{x_n}  \right\rangle
\end{equation*}

\paragraph{High Order Partial Derivatives}
If $f_{xy}$ exists and is continuous at $\bvec p$, then $f_{xy}(\bvec p) = f_{yx}(\bvec p)$.

\section{Differentiation Rules}
Let $f,g: \vecspace{R}^n \to \vecspace{R}$ and $a,b \in \vecspace{R}$.
\begin{align*}
    &\nabla( af + bg)  &=& a \nabla f + b \nabla g \tag{Linearity} \\
    &\nabla(fg) &=& f \nabla g + g \nabla f \tag{Product Rule} \\
    &\nabla(f/g) &=&\left(g\nabla f -  f \nabla g\right) / g^2 \tag{Quotient Rule}\\
    &\nabla (f^\alpha) &=& \alpha f^{\alpha - 1} \nabla f \tag{Power Rule}
\end{align*}
\paragraph{Chain Rule} Let $f: \vecspace{R}^n \to \vecspace{R}$, $f=f(\bvec{x})$, and $\bvec x: \vecspace{R}^p \to \vecspace{R}^n$ where $\bvec{x} = \bvec{x}(\bvec{t})$. Then
\begin{equation*}
    \frac{\partial f}{\partial t_i} = \nabla f \cdot \frac{\partial \bvec x}{\partial t_i}
\end{equation*}
E.g. for $f:\mathbb R^3 \to \mathbb R$, $\bvec x = \bvec x(s,t)$, \begin{equation*}
    f_s = f_x x_s + f_y y_s + f_z z_s
\end{equation*}


\section{Directional Derivative}
Consider $f: \vecspace{R}^n \to \vecspace{R}$ and $\bvec p, \bvec u \in \vecspace{R}^n$ with $\|\bvec u\|=1$. The directional derivative of $f$ in the direction of $\bvec u$ (or, with respect to $\bvec u$) at point $\bvec p$ is given by
\begin{equation*}
    D_\bvec{u} f(\bvec p) = \nabla f(\bvec p) \cdot \bvec u, \qquad D_{\bvec{e}_i} f = \pdiffop[f]{x_i}
\end{equation*}
Note that the directional derivative attains it maximum, $\|\nabla f(\bvec p)\|$, when $\bvec u$ is the unit vector in the direction of the gradient of $f$.

\section{Level Sets}
For a function $f: E \to \vecspace{R}$ where $E \subseteq \vecspace{R}^n$ and $k \in \vecspace R$, the level set where $f=k$  (or, the set containing $\bvec p$ where $f(\bvec p)=k$) is the set
\begin{equation*}
\left\{ \bvec{x} \in E \ \big\vert \ f(\bvec x) = k \right\}
\end{equation*}
\begin{itemize}[leftmargin=*,noitemsep]
    \item When $f:\mathbb R^2 \to \mathbb R$, the level set is called the \underline{``level curve"}.
    \item When $f:\mathbb R^3 \to \mathbb R$, the level set is called the \underline{``level surface"}.
    \item In general, the graph of a level surface of $f$ has one less dimensions that the graph of $f$.
    \item The gradient of $f$ is $\perp$ to the tangent line of a level set.
\end{itemize}

\paragraph{Tangent Set} The tangent set to the level set where $F=F(\bvec p)$, $\bvec p \in \mathbb R^n$, is given by \begin{equation*}
    \left\{  \bvec{x} \in \mathbb R^n \ \big\vert \ \nabla F(\bvec p) \cdot (\bvec x - \bvec p) = \bvec 0 \right\}
\end{equation*}
When $F: \begin{array}{c}\mathbb R^2 \\ \mathbb R^3\end{array}\to\mathbb R$, the tangent set is call the \underline{tangent}  $\begin{array}{c}\text{\underline{line}} \\ \text{\underline{plane}}\end{array}$

\paragraph{Critical Point} For $F: \mathbb R^n \to \mathbb R$, a critical point of $F$ is a point $\bvec p$ where either 1) $\nabla F(\bvec p) = \bvec 0$ or 2) F is not differentiable.

\section{Parameterizing Surfaces}
We frequently see surfaces as\begin{enumerate}[noitemsep]
    \item Graphs of functions $\mathbb R^2 \to \mathbb R$: z = f(x,y)
    \item Level surfaces of functions $\mathbb R^3 \to \mathbb R$: F(x,y,z) = c
\end{enumerate}
\paragraph{General Notion} Produce a function $\bvec r : D \to \mathbb R^n$ where $D \subset \mathbb R^{n-1}$ and refer to the image as the parameterized surface.
\paragraph{Regular Surfaces} A parameterization $\bvec r: D \to \mathbb R^3$, $D\subset \mathbb R^2$ is said to be (locally) regular at $\bvec p$ if 1) it is continuously differentiable (i.e. is $C^1$) at $\bvec  p$, and 2) all of the first order partial derivations of $\bvec r$ are linearly independent at $\bvec p$.

\subsection{Tangent Planes to Surfaces}
Consider the parameterization $\bvec r: D \to \mathbb R^3$, $D\subset \mathbb R^2$, with $\bvec p = \bvec{r}(u_0, v_0)$. The tangent plane to the surface at point $\bvec p$ is described by either
\begin{equation*}
(x,y,z) = \bvec{r}(u_0, v_0) + a\bvec{r}_u(u_0, v_0) + b\bvec{r}_v(u_0, v_0), \quad a,b\in\mathbb R
\end{equation*}
or, by letting $\bvec n = \bvec{r}_u \times \bvec{r}_v$,
\begin{equation*}
\bvec n \cdot (\bvec x - \bvec p) = \bvec 0
\end{equation*}

\section{Local Extrema}
\paragraph{Saddle Point} For $f: E \to \mathbb R$, $E\subset \mathbb R^n$, a saddle point is critical point $\bvec p$ at which $f$ is differentiable and $\nabla f(\bvec p) = \bvec 0$, which at which $f$ \emph{does not} attain a local extreme value.
\paragraph{Hessian Matrix} Suppose $f=f(x,y)$ is $C^2$ and $\bvec p \in \mathbb R^2$. The Hessian matrix of $f$ at $\bvec p$ is the matrix \begin{equation*}
    \matr H(\bvec p) = \left.\begin{bmatrix}
        f_{xx} & f_{xy} \\ f_{yx} & f_{yy}
    \end{bmatrix}\right|_\bvec{p}
\end{equation*}
\paragraph{Non-Degeneracy} A point $\bvec p$ is a non-degenerate critical point if $\nabla f(\bvec p) = \bvec 0$ and $\det \matr{H}(\bvec p) \neq 0$
\subsection{2nd Derivative Test}
Suppose $f=f(x,y)$ is $C^2$ and $\bvec p$ is a non-degenerate point of $f$. Then,
\begin{enumerate}[noitemsep]
    \item If $\det \matr{H}(\bvec p) > 0$ and $f_{xx}(\bvec p) > 0$, $f$ attains a local min.
    \item If $\det \matr{H}(\bvec p) > 0$ and $f_{xx}(\bvec p) < 0$, $f$ attains a local max.
    \item If $\det \matr{H}(\bvec p) < 0$, $f$ has a saddle point at $\bvec p$
\end{enumerate}

\section{Optimization}
To find all the global max/min values of a function $f=f(\vec x)$ on a compact set $D$:
\begin{itemize}[leftmargin=*,noitemsep]
    \item Find the values at the interior critical points
    \item Find the extreme values on the boundary of of $D$
    \item Note: No need to apply Hessian / 2nd Derivative Test
\end{itemize}

\newpage

\thispagestyle{fancy}
\cfoot{\footnotesize Copyright \textcopyright \ 2020 Brian Schubert}
\rfoot{\footnotesize Revised 2020-04-21}
\lhead{MATH 2321\\Calculus 3 for Science and Engineering}
\rhead{Spring 2020 \\ Dr. Sumi Seo}

\subsection{Lagrange Multipliers}

Given $f=f(\vec x)$, $g=g(\vec x)$ in $C^1$, a critical point subject to the constraint $g(\vec x) = c$ is a point $\vec x$ at which \begin{equation*}
    g(\vec x) = c \quad \text{and} \quad \nabla f(\vec x) = \lambda \nabla g(\vec x)
\end{equation*}
The scalar $\lambda$ is called a Lagrange Multiplier.

If $f$ attains a local max/min at $\vec p$, then $\vec p$ must be a critical point of $f$ subject to $g(\vec x) = c$

\section{Multivariable Integration}
\paragraph{Double Integrals}
\begin{equation*}
\iint_R f(x,y) \, \mathrm{d}A = \lim\limits_{n\to \infty} \sum_{k=1}^{n} f(\vec x_k)\Delta A, \quad \Delta A = \Delta x\Delta y
\end{equation*}

\paragraph{Fubini's Theorem} Suppose that $f=f(x, y)$ is continuous in the rectangle $R=[a,b]\times [c,d]$. The $f$ is Riemann integrable on $R$, and can be calculated via
\begin{equation*}
\iint_R f(x,y)\, \mathrm{d}A = \int_c^d \!\!\! \int_a^b\! f(x, y) \, \mathrm{d} x\, \mathrm{d}y = \int_a^b \!\!\! \int_c^d \! f(x, y) \, \mathrm{d} y\, \mathrm{d}x
\end{equation*}

\paragraph{Theorem} Let $R\subset \mathbb{R}^2$ be the region of points $(x,y)$ such that $x \in [a,b]$ and $p(x) \leq y \leq q(x)$. Then \begin{equation*}
    \iint_R f(x,y) \, \mathrm{d}A = \int_a^b \!\!\! \int_{p(x)}^{q(x)} \! f(x,y) \,\mathrm{d}y\,\mathrm{d}x
\end{equation*}

\paragraph{Theorem} If $R=R_1 \cup R_2 $, then \begin{equation*}
\iint_R f(x,y) \, \mathrm{d}A = \iint_{R_1} f(x,y) \, \mathrm{d}A + \iint_{R_2} f(x,y) \, \mathrm{d}A
\end{equation*}

\subsection{Polar Coordinates}

\begin{gather*}
\mathrm{d}A = r\,\mathrm{d}\theta\,\mathrm{d}r = r\,\mathrm{d}r\,\mathrm{d}\theta 
\shortintertext{where}
x = r\cos \theta, \quad y=r\sin\theta, \quad r \geq 0, \quad \theta \in [0, 2\pi)
\shortintertext{so}
\iint_R f(x,y)\,\mathrm{d}A = \iint_R f(r\cos\theta, r\sin\theta) \,r\,\mathrm{d}r\,\mathrm{d}\theta
\end{gather*}

\section{Integration in $\mathbb R^3$}

\paragraph{Triple Integral} The triple integral of $f=f(x,y,z)$ over the solid $S$ in $\mathbb R^3$ is
\begin{equation*}
\iiint_S\! f(x,y,z)\,\mathrm{d}V \!=\!\! \lim\limits_{n\to \infty} \sum_{k=1}^{n}\! f(\vec s_k)\Delta V,\! \begin{array}{c} \vec s_k =\! (x_k,\! y_k,\! z_k\!) =\! \text{``tiny box"} \\ \Delta V = \Delta x\Delta y\Delta z \end{array}
\end{equation*}
\paragraph{Theorem} If the solid $S$ is given by \begin{gather*}
p(x,y) \leq z \leq q(x,y), \quad u(x) \leq y \leq v(x),\quad a \leq x \leq b
\shortintertext{then}
\iiint_S f(x,y,z)\,\mathrm{d}V =  \int_a^b \!\!\! \int_{u(x)}^{v(x)} \!\!\! \int_{p(x,y)}^{q(x,y)} \! f(x,y,z) \,\mathrm{d}z\,\mathrm{d}y\,\mathrm{d}x
\end{gather*}

\subsection{Volume of a Solid}
\begin{equation*}
(\text{volume of $S$}) = \iiint_S \,\mathrm{d}V
\end{equation*}


\columnbreak

\begin{minipage}{\columnwidth}

\section{Cylindrical and Spherical Coordinates}
{% REALLY REALLY squish equations, beyond what \shortintertext offers
    \setlength{\belowdisplayskip}{-2pt}%
    \setlength{\abovedisplayskip}{0pt}%
    \paragraph{Cylindrical Coordinates}\begin{equation*}
    \mathrm{d}V = r\,\mathrm{d}z \,\mathrm{d}r\,\mathrm{d}\theta
    \end{equation*}where\begin{equation*}
    x = r\cos \theta, \quad y=r\sin\theta, \quad z=z
    \end{equation*}so\begin{equation*}
    \iiint_R f(x,y,z)\,\mathrm{d}V = \iiint_R f(r\cos\theta, r\sin\theta, z) \,r\,\mathrm{d}z\,\mathrm{d}r\,\mathrm{d}\theta
    \end{equation*}
}

\paragraph{Spherical Coordinates}
{% REALLY REALLY squish equations, beyond what \shortintertext offers
    \setlength{\belowdisplayskip}{-2pt}%
    \setlength{\abovedisplayskip}{0pt}%
    \begin{equation*}
    \mathrm{d}V = \rho^2 \sin \phi \,\mathrm{d}\rho \,\mathrm{d}\phi\,\mathrm{d}\theta 
    \end{equation*}where\begin{gather*}
    x = \rho \sin\phi \cos \theta , \quad y=\rho \sin \phi \sin\theta, \quad z= \rho \cos \phi \\
    (r=\rho \sin \phi, \quad r^2 + z^2 = x^2 + y^2 + z^2 = \rho^2)
    \end{gather*}so\begin{equation*}
    \iiint_R f(x,y,z)\,\mathrm{d}V = \iiint_R f(r\cos\theta, r\sin\theta, z) \,r\,\mathrm{d}z\,\mathrm{d}r\,\mathrm{d}\theta
    \end{equation*}
}

\section{Density and Mass}

Suppose that an (ideal) 2-dimensional plate occupies a region $R\subset \mathbb R^2$, and that $\delta_{\mathrm{ar}}=\delta_{\mathrm{ar}}(x, y)$ gives the area-density (mass per unit area) at each point. Then, an infinitesimal piece of mass (an element of mass) is $\mathrm{d}m = \delta_{\mathrm{ar}}(x,y)\,\mathrm{d}A$. Consequently, 
\begin{gather*}
    (\text{mass}) = \iint_R \mathrm{d}m = \iint_R \delta_{\mathrm{ar}}(x,y)\,\mathrm{d}A
\shortintertext{Likewise, for $R\subset \mathbb R^3$}
    (\text{mass}) = \iiint_S \mathrm{d}m = \iiint_R \delta(x,y,z)\,\mathrm{d}V
\end{gather*} 

\section{Surface Area}
Given a (basic) parameterization $\vec r: D \to \mathbb R^3$, $\vec r = \vec r(u,v)$, for a surfae, an \underline{element of area} on the surfae is given by \begin{gather*}
\mathrm{d}S = \|\vec r_u \times \vec r_v \| \,\mathrm{d}u\,\mathrm{d}v
\shortintertext{and the surface area pf the surface parameterized by $\vec r$ is}
(\text{SA}) = \iint_D \mathrm{d}S = \iint_D \|\vec r_u \times \vec r_v \| \,\mathrm{d}u\,\mathrm{d}v 
\end{gather*}

\paragraph{Surface Area of Graph}
In the special case when the parameterized surface is given by the graph of a function, i.e. $z=f(x,y)$, we have \begin{equation*}
\vec r(u,v) = (u,v,f(x,y))
\end{equation*}

If $f$ is continuously differentiable, then so is $\vec r$.

The graph $M$ of $f: D\to \mathrm R$ is a surface with a basic parameterization $\vec r :D \to \mathrm R^3$ given by
\begin{gather*}
\vec r_u \times \vec r_v = (1,0, f_x) \times (0, 1, f_y) = (-f_x, -f_y, 1)
\shortintertext{so the element of area is}
\mathrm{d}S = \|\vec r_u \times \vec r_v \| \,\mathrm{d}u\,\mathrm{d}v = \sqrt{f_x^2 + f_y^2 + 1}\mathrm{d}y\,\mathrm{d}x
\shortintertext{and the surface area of $M$ is}
\iint_D =  \sqrt{f_x^2 + f_y^2 + 1}\,\mathrm{d}y\,\mathrm{d}x
\end{gather*}
\end{minipage}

\columnbreak


\begin{minipage}{\columnwidth}
\section{Vector Fields}
A vector fields on an open set $U \subset \mathbb{R}^n$ is a function $\vec F: U \to \mathbb{R}^n$ where for each $\vec p \in U$, $\vec F(\vec p)$ is ``considered as a vector based at $\vec p $''. I.e. a vector field is a function which assigns a vector to every point in $U$.

\paragraph{Conservative VF}
Given $f=f(\vec{x})$, $f:\mathbb{R}^n \to \mathbb{R}$
\begin{itemize}[leftmargin=*,noitemsep]
    \item $\nabla f$ is a gradient vector field.
    \item $\vec{F}=\nabla f$ is called a conservative vector field.
    \item $f$ is called a potential function.
\end{itemize}
$\vec{F}$ is conservative if and only if $\nabla\times\vec{F}=\vec{0}$.

\paragraph{Divergence} Given $\vec{F}=\vec{F}(\vec{x}) = \big\langle F_1 (\vec{x}), F_2 (\vec{x}), \ldots, F_n (\vec{x}) \big\rangle$,
\begin{equation*}
\operatorname{div} \vec{F} = \nabla\cdot\vec{F}= \pdiffop[F_1]{x_1} + \pdiffop[F_2]{x_2} + \cdots + \pdiffop[F_n]{x_n} = \sum_{i=1}^n \pdiffop[F_i]{x_i}
\end{equation*}

\paragraph{Curl} Given $\vec{F}(\vec{x}) = \big\langle P(x,y,z), Q(z,y,z), R(x,y,Z) \big\rangle$,
\begin{equation*}
\operatorname{curl} \vec{F} = \nabla \times \vec{F} = \begin{vmatrix}
\hat{i} & \hat{j} & \hat{k} \\
\pdiffop{x} & \pdiffop{y} & \pdiffop{z} \\
P & Q & R
\end{vmatrix} = \big\langle R_y - Q_z,  - (R_x - P_z),  Q_x - P_y\big\rangle
\end{equation*}
For $\vec{F}: \mathbb{R}^2 \to \mathbb{R}^2$, $\operatorname{curl}\vec{F} = Q_x - P_y$.

\section{Line Integrals}
Suppose $\vec{F}$ is a continuous VF on $\mathbb{R}^n$, $\vec{r}: [a,b] \to \mathbb{R}^n$ is a regular parameterization of a curve, and $C$ is the oriented curve from $\vec{r}(a)$ to $\vec{r}(b)$ defined by $\vec{r}$. Then the line integral of $\vec{F}$ along $C$ is
\begin{equation*}
\int_C \vec{F}\cdot\mathrm{d}\vec{r} = \int_a^b \vec{F}(\vec{r}(t))\cdot \vec{r'}(t)\,\mathrm{d}t
\end{equation*}
If $\vec{F}$ is a force field, $\int_C \vec{F}\cdot\mathrm{d}\vec{r}$ is the work done by $\vec{F}$ along $C$.
\paragraph{Fundamental Theorem of Line Integral}
Let $\vec{F}$ be a conservative VF with potential function $f$, and $C$ be an oriented curve from $\vec a$ to $\vec b$ defined by $\vec{r}: [t_0, t_1]\to \mathbb{R}^n$ with $\vec{r}{t_0} = \vec a$ and $\vec{r}(t_1)=\vec b$. Then,
\begin{equation*}
\int_C \vec{F}\cdot\mathrm{d}\vec{r} = f(\vec{b}) -f(\vec{a})
\end{equation*}

\paragraph{Green's Theorem}
\begin{itemize}[leftmargin=*,noitemsep]
    \item A simple closed curve in $\mathbb{R}^n$ is a closed curve that does not intersect itself.
    \item Let $R$ be a compact region bounded by the simple closed curve $C$. We say that $\partial R = C$ is given the positive orientation if it is oriented counter-clockwise.
\end{itemize}
Let $R$ be a region in $\mathbb{R}^2$ and $\partial R$ be the positive oriented boundary. Let $\vec{F}=(P, Q)$ be a VF in $\mathbb{R}^2$. Then \begin{equation*}
\int_C \vec{F}\cdot\mathrm{d}\vec{r} = \int_{\partial R} \vec{F}\cdot\mathrm{d}\vec{r} = \iint_R \left(Q_x - P_y\right)\,\mathrm{d}A.
\end{equation*}
\end{minipage}

\newpage

\thispagestyle{fancy}
\cfoot{\footnotesize Copyright \textcopyright \ 2020 Brian Schubert}
\rfoot{\footnotesize Revised 2020-04-21}
\lhead{MATH 2321\\Calculus 3 for Science and Engineering}
\rhead{Spring 2020 \\ Dr. Sumi Seo}

\begin{minipage}{\columnwidth}
% Extra squish all new entries
%\setlength{\belowdisplayskip}{-2pt}%
%\setlength{\abovedisplayskip}{2pt}%

\section{Flux Through a Surface}
\paragraph{Notation}Let $M$ be the surface parameterized by $\vec{r}: D\to\mathbb{R}^3$, $\vec{V}$ be a VF representing the flow of fluid (flux), and $\vec{n}$ be a unit normal vector to each point of $M$.
\paragraph{Oriented Surface} A surface $M$ is oriented if $M$ has two sides and one of them has been designed to be the positive side and it is possible to choose a unit normal vector $\vec{n}$ at every point so that $\vec{n}$ varies continuously over $M$. The given choice of $M$ provides $M$ with an orientation.
\paragraph{Closed Surface} A closed surface is a surface which is the boundary of a solid region. E.g. a sphere is the boundary of a ball.
\paragraph{Flux} The flux of a fluid through a surface is determined by the component of $\vec{V}$ that is in the direction of $\vec{n}$, i.e. $\vec{v}\cdot\vec{n}$:
\begin{gather*}
(\text{flux of $\vec{V}$ over $M$}) = \iint_M \vec{V}\cdot\vec{n}\,\mathrm{d}S
\shortintertext{where}
\vec{n}=\frac{\vec{r}_u \times \vec{r}_v}{\|\vec{r}_u \times \vec{r}_v \|}, \quad \mathrm{d}S = \| \vec{r}_u \times \vec{r}_v\|\mathrm{d}u\,\mathrm{d}v
\end{gather*}

\subsection{Divergence Theorem}
Suppose $M$ is a closed surface, $E$ is a solid region in $\mathbb{R}^3$ with $M=\partial E$ that is oriented outward, and $\vec{F}$ is a VF in $\mathbb{R}^3$. Then,
\begin{equation*}
(\text{flux}) = \iint_M \vec{F}\cdot\vec{n}\,\mathrm{d}S = \iint_{\partial E} \vec{F}\cdot\vec{n}\,\mathrm{d}S = \iiint_E \operatorname{div}\vec{F}\,\mathrm{d}V
\end{equation*}

\subsection{Stoke's Theorem}
\begin{itemize}[leftmargin=*,noitemsep]
    \item For calculating line integral along closed curve in $\mathbb{R}^3$ 
    \item ``3D analog of Green's Theorem''.
\end{itemize}

Document incomplete.



\end{minipage}
\end{multicols}
\end{document}