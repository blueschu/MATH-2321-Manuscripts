\documentclass[10pt,landscape,letterpaper]{article}
\usepackage{multicol}
%\usepackage{calc}
%\usepackage{ifthen}
\usepackage[landscape]{geometry}
\usepackage{amsmath,amsfonts,amssymb}
%\usepackage{color,graphicx,overpic}
%\usepackage{hyperref}
%\usepackage{tabularx}
\usepackage{fancyhdr}
\usepackage{enumitem}
\usepackage{tabularx}
\usepackage{titlesec}
\usepackage{array}
% For Lin Alg section
%\usepackage{tikz}
%\usetikzlibrary{calc}
%\usepackage{mathtools}

%
% Styling inspired by tex.stackexchange user "Dror" in this posting:
% https://tex.stackexchange.com/questions/8827/preparing-cheat-sheets
%

\thispagestyle{fancy}
\cfoot{\footnotesize Copyright \textcopyright \ 2020 Brian Schubert}
\rfoot{\footnotesize Revised 2020-02-12}
\lhead{MATH 2321\\Calculus 3 for Science and Engineering}
\rhead{Spring 2020 \\ Prof. Sumi Seo}
%\chead{\large \underline{Midterm Exam ``Cheat Sheet"}}
\renewcommand{\headrulewidth}{0pt}
\setlength{\voffset}{-4pt}

\geometry{top=0.7in,left=.4in,right=.4in,bottom=.6in}


% Redefine section commands to use less space
\makeatletter
\renewcommand{\section}{\@startsection{section}{1}{0mm}%
                                {-1ex plus -.5ex minus -.2ex}%
                                {0.5ex plus .2ex}%x
                                {\normalfont\large\bfseries}}
\renewcommand{\subsection}{\@startsection{subsection}{2}{0mm}%
                                {-1explus -.5ex minus -.2ex}%
                                {0.5ex plus .2ex}%
                                {\normalfont\normalsize\bfseries}}
% Custom - reduce font size of paragarphs
\titleformat*{\paragraph}{\normalfont\small\bfseries}
%\renewcommand{\subsubsection}{\@startsection{subsubsection}{3}{0mm}%
%                                {-1ex plus -.5ex minus -.2ex}%
%                                {1ex plus .2ex}%
%                                {\normalfont\small\bfseries}}

% Don't print section numbers
\setcounter{secnumdepth}{0}


\setlength{\parindent}{0pt}
\setlength{\parskip}{0pt plus 0.5ex}

% Notation Definitions
\newcommand{\dd}[1]{\mathrm{d}#1}
\newcommand{\diffop}[2][]{\frac{\dd#1}{\dd #2}}
\newcommand{\pdiffop}[2][]{\frac{\partial #1}{\partial #2}}
\newcommand{\matr}[1]{\mathbf{#1}}
\newcommand{\bvec}[1]{\mathbf{#1}}
\newcommand{\vecspace}[1]{\mathbb{#1}}
\newcommand{\transp}{^\mathrm{T}}

\newcommand\cheatsheetmargin{0.2cm}

% -----------------------------------------------------------------------

\begin{document}
\raggedright
\footnotesize
\begin{multicols}{3}


% multicol parameters
% These lengths are set only within the two main columns
%\setlength{\columnseprule}{0.25pt}
\setlength{\premulticols}{1pt}
\setlength{\postmulticols}{1pt}
\setlength{\multicolsep}{1pt}
\setlength{\columnsep}{2pt}

%\begin{center}
%     \Large{\underline{Preliminaries}} \\
%\end{center}
\section{Geometry in $\mathbb R^n$}
\paragraph{Orthogonal Projection \& Vector Product Identities} \begin{gather*}
    \operatorname{proj}_\bvec{v} \bvec{F} = \left(\frac{\bvec F \cdot \bvec v}{\bvec v \cdot \bvec v}\right) \bvec v = \left(\frac{\bvec F \cdot \bvec v}{\|\bvec v\|^2}\right) \bvec v \\
    \bvec v \cdot \bvec w = \|\bvec v\|\|\bvec w\| \cos\theta = \|\bvec v\| \|\operatorname{proj}_\bvec{v} \bvec w\|  = \|\bvec w\| \|\operatorname{proj}_\bvec{w} \bvec v \| \\
    \bvec v \times \bvec w = -( \bvec w \times \bvec v), \quad \| \bvec v \times \bvec w\| = \|\bvec v\|\|\bvec w\| \sin\theta
\end{gather*}
\subsection{Planes} A plane in $\mathbb R^m$ which contains the points $\bvec p=\langle p_1,\ldots,p_m \rangle$ and normal to the vector $\bvec n = \langle n_1,\ldots,n_m \rangle$ is given by
\begin{gather*}
\bvec{n}\cdot (\bvec x - \bvec p) = \bvec 0 \iff n_1(x_1 - p_1) + \cdots + n_m(x_m - p_m) = 0\tag{basic form}\\
n_1 x_1 + \cdots + n_m x_m + d = 0, \quad d=-n_1 p_1 - \cdots -n_mp_m \tag{standard form}
\end{gather*}
\subsection{Lines}A line $\ell$ in $\mathbb R^n$ containing the point $\bvec p$ and is parallel to the vector $\bvec v$ is given by
\begin{gather*}
    \bvec x(t) = \bvec p + t\bvec v \tag{Vector Equation}\\
    {x}_1 = p_1 + v_1 t, \ \ldots,\ {x}_n = p_n + v_n t \tag{Parametric Equation}\\
    \frac{x_1 - p_1}{v_1} = \cdots = \frac{x_n - p_n}{v_n} \tag{Symmatric Equation}
\end{gather*}
\paragraph{Tangent Line} Given $\bvec p: \mathbb R \to\mathbb R^n$, the tangent line $L$ at $t=t_0$ is parameterized by $L(r) = \bvec p(t_0) + r \bvec p'(t_0)$.


\section{Basic Topology of $\vecspace{R}^n$}
\paragraph{Open Ball} The open ball centered at $\bvec p \in \vecspace R^n$ of radius $r$ is
\begin{equation*}
    B_r^n (\bvec p) = \left\{\bvec x \in \vecspace{R}^n \ \vert \ \operatorname{dist} (\bvec x, \bvec p) < r \right\}
\end{equation*}


\paragraph{Open Set}

\section{Multivariable Derivatives}
Given $f: \vecspace{R}^n \to \vecspace{R}$ where $f = f(\bvec x)$, the partial derivative of $f$ with respect to the $i$th component of $\bvec x$ is given by
\begin{equation*}
    f_{x_i}(\bvec p) = \left.\frac{\partial f}{\partial x_i}\right|_\bvec{p}
        = \lim\limits_{h \to 0} \frac{f(\bvec p + h\hat{\mathbf{e}}_i)-f(\bvec p)}{h}
\end{equation*}

\paragraph{Gradient} For the function $f$ above, the gradient of $f$, $\nabla f: \vecspace{R}^2 \to \vecspace{R}^2$ is given by
\begin{equation*}
    \nabla f = \left\langle \pdiffop[f]{x_1}, \pdiffop[f]{x_2}, \ldots, \pdiffop[f]{x_n}  \right\rangle
\end{equation*}

\paragraph{High Order Partial Derivatives}
If $f_{xy}$ exists and is continuous at $\bvec p$, then $f_{xy}(\bvec p) = f_{yx}(\bvec p)$.

\section{Linearization \& Linear Approximation}
For $f: \vecspace{R}^n \to \vecspace{R}$, the linearization of $f$ about point $\bvec{p} \in \vecspace{R}^n$ is
\begin{equation*}
    L_f(\bvec{x};\, \bvec{p}) = f(\bvec{p}) + \nabla f(\bvec p) \cdot (\bvec x - \bvec p)
\end{equation*}
\paragraph{Linear Approx.} $f(\bvec x) \approx L(\bvec{x};\, \bvec p)$ when $\bvec{x}$ near $\bvec p$.

\section{Differentiation Rules}
Let $f,g: \vecspace{R}^n \to \vecspace{R}$ and $a,b \in \vecspace{R}$.
\begin{align*}
    &\nabla( af + bg)  &=& a \nabla f + b \nabla g \tag{Linearity} \\
    &\nabla(fg) &=& f \nabla g + g \nabla f \tag{Product Rule} \\
    &\nabla(f/g) &=&\left(g\nabla f -  f \nabla g\right) / g^2 \tag{Quotient Rule}\\
    &\nabla (f^\alpha) &=& \alpha f^{\alpha - 1} \nabla f \tag{Power Rule}
\end{align*}
\paragraph{Chain Rule} Let $f: \vecspace{R}^n \to \vecspace{R}$, $f=f(\bvec{x})$, and $\bvec x: \vecspace{R}^p \to \vecspace{R}^n$ where $\bvec{x} = \bvec{x}(\bvec{t})$. Then
\begin{equation*}
    \frac{\partial f}{\partial t_i} = \nabla f \cdot \frac{\partial \bvec x}{\partial t_i}
\end{equation*}
E.g. for $f:\mathbb R^3 \to \mathbb R$, $\bvec x = \bvec x(s,t)$, \begin{equation*}
    f_s = f_x x_s + f_y y_s + f_z z_s
\end{equation*}


\section{Directional Derivative}
Consider $f: \vecspace{R}^n \to \vecspace{R}$ and $\bvec p, \bvec u \in \vecspace{R}^n$ with $\|\bvec u\|=1$. The directional derivative of $f$ in the direction of $\bvec u$ (or, with respect to $\bvec u$) at point $\bvec p$ is given by
\begin{equation*}
    D_\bvec{u} f(\bvec p) = \nabla f(\bvec p) \cdot \bvec u, \qquad D_{\bvec{e}_i} f = \pdiffop[f]{x_i}
\end{equation*}
Note that the directional derivative attains it maximum, $\|\nabla f(\bvec p)\|$, when $\bvec u$ is the unit vector in the direction of the gradient of $f$.

\section{Level Sets}
For a function $f: E \to \vecspace{R}$ where $E \subseteq \vecspace{R}^n$ and $k \in \vecspace R$, the level set where $f=k$  (or, the set containing $\bvec p$ where $f(\bvec p)=k$) is the set
\begin{equation*}
\left\{ \bvec{x} \in E \ \big\vert \ f(\bvec x) = k \right\}
\end{equation*}
\begin{itemize}[leftmargin=*,noitemsep]
    \item When $f:\mathbb R^2 \to \mathbb R$, the level set is called the \underline{``level curve"}.
    \item When $f:\mathbb R^3 \to \mathbb R$, the level set is called the \underline{``level surface"}.
    \item In general, the graph of a level surface of $f$ has one less dimensions that the graph of $f$.
    \item The gradient of $f$ is $\perp$ to the tangent line of a level set.
\end{itemize}

\paragraph{Tangent Set} The tangent set to the level set where $F=F(\bvec p)$, $\bvec p \in \mathbb R^n$, is given by \begin{equation*}
    \left\{  \bvec{x} \in \mathbb R^n \ \big\vert \ \nabla F(\bvec p) \cdot (\bvec x - \bvec p) = \bvec 0 \right\}
\end{equation*}
When $F: \begin{array}{c}\mathbb R^2 \\ \mathbb R^3\end{array}\to\mathbb R$, the tangent set is call the \underline{tangent}  $\begin{array}{c}\text{\underline{line}} \\ \text{\underline{plane}}\end{array}$

\



text

text

text

text

text

text

text

text

text

text

text

text

text

text

text

text

text

text

text

text

text

text

text

text

text

text

text

text

text

text

text

text

text

text

text

text

text

text

text

text

text

text

text

text

text

text

text

text

text

text

text

text

text

text

text

text

text

text

text

text

text

text

text

text

text

text

text

text

text

text

text

text

text

text

text

text

text

text

text

text

text

text

text

text

text

text

text

text

text

text

text

text

text

text

text

text

text

text

text

text

text

text

text

text

text

text

text

text

text

text

text

text

text

text

text

text

text

text

text

text
\end{multicols}
\end{document}